\documentclass[13ptletterpaper]{paper}
%this is where the images come from
\usepackage{graphicx}
\usepackage{setspace}
\usepackage[english]{babel}
\graphicspath{ {images/} }
%tab command
\newcommand\tab[1][1cm]{\hspace*{#1}}
\setlength{\oddsidemargin}{-0.25in} % Left margin of 1 in + 0 in = 1 in
\setlength{\textwidth}{7in}   % Right margin of 8.5 in - 1 in - 6.5 in = 1 in
\setlength{\topmargin}{-.75in}  % Top margin of 2 in -0.75 in = 1 in
\setlength{\textheight}{9.2in}  % Lower margin of 11 in - 9 in - 1 in = 1 in
\doublespacing
% --- create title format ---
\title{
	\begin{center}
		\normalfont \normalsize
		\rule{\linewidth}{.5pt} \\[0.4cm] 
		\huge {The relationship Between the Digital and Physical world} \\ 
		\large {Master's Project}\\
		\small{Chris P David}\\
		{\today}\\
		\rule{\linewidth}{.5pt} \\
	\end{center}
}
% --- end title format ---


\begin{document}
	\begin{titlepage}
		\clearpage
		\maketitle
		\thispagestyle{empty}
	\end{titlepage}
	\pagebreak	
	\tableofcontents
	
	\begin{flushleft}
		\pagebreak
		\section{Proposal}
		\tab For my Master’s project I will be creating a Robot with a Raspberry Pi 3b computer	board and essentially be creating a robot that listens to the user’s input. Overall this project is going to convert the digital signal that the user inputs and have the robot output physical movement. The robot will simulate the relationship between digital and physical I/O that users interact with in daily life. This robot will be enclosed into a maze and the user will need, via application on the phone, to tell the robot how to move. On the application there will be options to see what the	robot is looking at (via mounted camera on robot) as well as buttons to allow the robot to move	in the basic 4 directions.	\\
		\tab The robot will “listen” or obtain signals via the Wi-Fi dongle on the board from the user’s phone. In this case the robot will be listening to the movement buttons on the application. After getting a input from the on-screen buttons, the application will convert button press to a raspberry pi command that will in turn cause a physical change on the board. In this case it will turn a server motor on or off essentially.I will be, after getting the robot to fluidly move, add another robot to “chase” the player controlled robot. This robot will be following commands off a database, random movements or	cloning the player’s movements. This will be done in order to fulfill the database portion of the Master’s Project guidelines (unless told that I need to think of something else). I will be programming in the Raspberry Pi environment, in which, I will develop an android based application that will be the “controller” for the robot. \\
		\tab If both portions of the project are done successfully, the last portion will be to develop a	VR simulation for seeing what the robot sees via the Raspberry Pi camera that is attached to the robot. Using the Google VR SDK, along with a google cardboard, I should be able to allow users	to get a first-person view of what their robot sees. In conclusion hopefully the robot will exemplify the relationship between digital, electronic signals that are created by the code, generated by the user button press, and have it translate into physical signals that affect physical components.
		\subsection{Comments about proposal after completion of Project}
		\tab After completing the project several things were changed such as the ability for the application, the other robot and lastly using the camera. The reason for this was that I simply had not enough time to get all of these options to work, it was difficult enough getting one robot to work. I tried to figure out how would the camera work with the current build and it can't at the moment simply because of conflicting library that I'm using namely Pygame. On the other hand I worked upon a new topic that I didn't touch with the proposal which was game programming.
		\section{Plan of Study}
		\tab Essentially I’m thinking of doing a majority of the mechanical portions of my project before my software requirements.
		During the month of September, I plan to create the robot, but also at the same time see if there’s any other improvements I’d like to do and make notes on it. I also plan to get the robot up and running and see both speed and turning abilities. Also during this time, I can make notes on how the “game field” should look like.\\
		\tab October will have to be part 1 of developing the software. I need to figure out firstly how communications will be between the software and the robot so a simple test or going forward, left, right and backwards seems to be good. I will also plan here if for the movement of the robot if going backwards is better or if turning the robot 360 degrees would be better. I also plan to see if I’m able to make a rudimentary website/game page to at least make the robot follow my control inputs as well as see how the final game should look like.\\
		\tab November will be part 2 of developing the software as all the kinks from mechanical should be done at this point. I would need to polish and build on the web game interface as well as see how the Raspberry Pi camera would work probably by overlaying the camera feed over the controls or better yet just have them always on providing a cool first person experience. \\
		\tab December will be the final phase with me working on the write-up, collecting all my research notes and basically getting ready for the presentation. I also plan to see if I can get any last minute additions onto the robot or software. I will also do some more stress testing (this will be done in every stage), but it’s honestly just to have a final stress test, for movement, software, robot controllability, visuals, and also even battery so I can see if I would need extra batteries for the presentation. \\
		\tab Overall a lot of this project depends if mechanical parts don’t fail as well as the software not encountering some hiccups but it should be thought of as there will be thus when issues arise work on them immediately.\\
		
		\subsection{Gannt Graph}
		\includegraphics[scale=0.75]{Capture.png}
		\section{Working though the Project}
		\subsection{September}
		\tab During this month I did the initial steps along with getting all the parts. Simply put I got the robot kit as well as the Raspberry Pi. I soldered and constructed the robot and was able to get it to work. I was also able to get the Raspberry Pi install to work as well, putting Raspbian onto it. There were many other options such as Linux, XBMC, OpenElec, RetroPie but I'm not making a media center nor a emulator as well as that this is my frist time working with a Pi so I went with Raspbian. During this time I also ran into what is the long list of issues that have plagued me during this Project.
		\subparagraph{Issues in September} After running the Robot for the first test, I noticed that the motor on the left side wasn't working nor was the fact that no power seemed to be going to my pi. I looked up to the Specifications regarding the Pi and saw that in order for the Pi to remain powered it needs at least 4.6A to be on, so I checked what could be wrong and lo and behold it was my batteries that I had, simply put they were too cheap! So once that was fixed I went and see what I could do for the next issue. Which was dealing with the weird motor. Now in my opinion it had to do to two things. \\
		\tab 1. This was my frist time soldering in almost 3 years so I was a bit rusty.\\
		\tab 2. The parts that where given where not the greatest of quality so I was limited to that.\\
		I had to buy spare motor to get it to work which took 3 weeks to get to here. This impacted my expected time lines by 2 weeks but not too much as I prepared for this to happen. \\
		
		\subsection{October} 
		\tab In this month I was expecting to get input down and start the website, this is where I had my change occur which was to instead of focus on the website create a game in which I controlled the robot. The reason I needed to do this was that I encounter the issue in regards to operating the robot via wireless communication. I figured that focusing on creating a program that works locally on the robot would be much better than relying on a site that may not work due to security restrictions of the campus's network. 
		\subparagraph{Issues in October} But this is also where I encountered a main issue and that was in the form of a burnt batter pack. I was notified of this when I smelt a burning of plastic smell in my room coming home one day. I still don't know how this happened but this cost me another week and not only that but it was when I was trying to get my input controls down, nevertheless the shifted towards the game portion to figure that out and then deal with the batter pack issue.
		\subsection{November} 
		\tab During this month I worked on the game which I modeled off of a basic maze type game that is the tutorial into programming much more complex games. So I figured out how to make a maze and at first I created it with using a text file to output the maze but this in turned had some issues with regards of sprite animation in which it worked perfectly but the player image was not correctly displaying the player but also that it wasn't moving. So I decided to redo the maze as an array, something I've done in Java before but not in Python. This is done with conjunction of a tutorial of maze creation but in the end I was able to figure out how to alter the seed to give a new maze as well as alter dimensions to alter the complete shape of the maze. I also figured out how to get player input in terms of a player name for the game.
		\subparagraph{Issues in November}Like before the main issues where the game program and as well as the battery pack. 
		\subsection{December} 
		\tab December came though and I was able to present the robot to Professor O'Rourke. Now two important thing happened in the following meetings. I figured that I had an issue with how the robot moved as well as wires. Wired was in the way I had to actually present the robot. I needed to hook up not only an HDMI to the pi but 2 USB's for mouse and keyboard. This not only made the setup clunky but also affected the performance of the robot itself. It was during this time Vince DeRusso was able to figure out the solution to my connection issue and that was to connect to another network due to security issues. 
		\subparagraph{Issues in December} Now the issue for the movement of the robot was twofold, It would go the wrong way plus the turning was very,very erratic. At first I did a neat way for the motor to basically accept which way the heading is currently facing then based on that input from the user, move in the way the sprite moves. This is known as Tank-based movement, it's based off of the Atari. This control layout is the idea of basing the controls on the player not the camera in that if a player is facing south, forward is south, backward is north, left is east , right is west, etc. Using this movement scheme helped me programing the robot to move effortlessly in conjunction with the player sprite on screen. The other issue was the movement of the robot itself in-terms of moving with the player sprite when I finally combined the two. The original method of using , in Python this is called a Dictionary switch statement, would not work at all as after the player indicated which direction to move, it would be stuck in the loop of moving the robot, thus the player would not be able to move. So I created another method in which two boolean values are done to essentially recreate the tank based movement. 
		\subsection{Finale} 
		\tab After getting all the parts to work, acquiring the WiFi, and finally the movement down I was done with this project. Everything worked and the robot seems to work as intended. There's still is an issue with the turning radius of the robot itself but that's something I will comment on in the Conclusion. The robot will move with the sprite (slow I may add as I encountered an issue where giving too much power caused the robot to move wierd) and once at the end it finishes! 
		
		\section{Conclusion}
		\tab This project taught me alot about the two world of both physical and digital. As an Computer Science student often times I'm mostly worried about the digital world, in terms of if this portion of the code doesn't work then it's something I can fix. I'm not either worrying about memory because the computer is already doing that for me, but something I don't really notice is that I'm never caring about the physical memory of that computer. I don't worry about the age of the HDD nor the age of SSD. I'm neither worrying about the processor because as a CS student I'm rather worrying about the efficiency of my code, not realizing that hardware almost plays a factor into it as well. \\
		\tab We often overlook how important the symbioses between digital (code) is with the physical(hardware). Oftentimes when we work only with code we worry only about the errors within it rather if anything is wrong with the hardware portion of that product or software. This project helped me see that in a much better light and helped me become much more aware of it. It also guided me to see how the two interact with another.\\
		\tab For this project I saw how the limitations of the robot prevented me to have complete prefect control over it likewise the actual software I used to code the program didn't interact very well with the camera and the motor's at sometimes.
	\end{flushleft}	
\end{document}